\vspace{5pt} \hrule \vspace{5pt}

\chapter{FA}

\section{拓扑线性空间}

\subsection{Topology of TLS}
四菜一汤:代数结构(线性结构,群环域),拓扑结构(拓扑,度量,范数,内积),测度结构(可积先可测),序结构。

TLS:线性和拓扑空间,且加法和数乘(作为乘积拓扑空间到拓扑空间的映射)连续,即线性结构与拓扑结构相容。

1. $T_{\lambda, a}: X \to X, T_{\lambda, a}(x) = \lambda x + a, \lambda \in \mathbb{K}, a \in X$是拓扑同胚。

2. TLS is T2 and T3.

邻域:包含(包含点的)开集。邻域全体$\mathcal{N}(x)$。

邻域基:每个邻域都能找到更小的邻域基中的元素(仍为邻域)含于其中。(这里不需要说是谁的邻域基,因为TLS中有线性结构,因此有平移操作,任意一点处的邻域基可以由零点处的邻域基所生成,类似Lie群)考虑零点的邻域基$\mathcal{U}(\theta)$,则$\forall a \in X$的邻域可以表示成$a + U, U \in \mathcal{U}(\theta)$,进一步$V + U = \cup_{a \in V} (a + U)$是开的。两个TLS相同,若其邻域基相容,即邻域基之于TLS相当于拓扑基之于TS,事实上邻域基完全决定了TLS的拓扑。

NLS的零点的邻域基为单点基$\{ B(0, 1) \}$。(这是因为有范数)以这个单位球为起点,定义:

(a). 吸收集:$\forall x \in \textbf{X}, \exists \alpha \textbf{>} 0 \text{ s.t. } \forall |\lambda| \le \alpha, \lambda x \in S$(包含\textbf{X中}任意一点确定的仿射线段的某个伸缩)

(b). 平衡集:$\forall s \in \textbf{S}, \forall \alpha \in \mathbb{K}, |\alpha| \le 1 \text{ s.t. }  \alpha s \in S$(包含\textbf{S中}任意一点确定的仿射线段)平衡包$\bar{S}^b = \{ \alpha s | |\alpha| \le 1, s \in S \}$。(包含S的最小闭球)

(c). 凸集:$\forall \alpha \in [0, 1], \alpha S + (1 - \alpha) S \subset S$。(主打一个入乡随俗)平衡凸 = 平衡 + 凸。凸包$\bar{S}^c$。(包含S的最小凸集)平衡凸包$\bar{S}^{bc} = \overline{\bar{S}^b}^c \neq \overline{\bar{S}^c}^b$。(包含S的最小平衡凸集)(一个例子:考虑x轴和y轴正半轴构成的空间,先取平衡包是整个x轴和y轴,再取凸包是全空间,先取凸包是第一象限,再取平衡包是一三象限)

注:关于数乘,在实数域表示伸缩,在复数域表示旋转和伸缩,所以在讨论问题的时候必须先说好是实数域还是复数域,在实数域用旋转和伸缩就乱套了,事实上上面很多就是,你说的闭球什么的就都是错的。旋转是说复数域里面的伸缩放到实数域里面看有旋转的效果,这个定义想说的还是伸缩下的性质,就是仿射性质。写这个注是受到一个例子的启发。

平衡集但不是凸集的例子:考虑实数域:$\{(x, y) \in \mathbb{R}^2 | xy = 0\}$,即x轴加y轴,显然是平衡的,不是凸的。考虑复数域把$\mathbb{R}$换成$\mathbb{C}$就好了,那就是$\mathbb{C}^2 \cong \mathbb{R}^4$了。

于是我们只需要考虑实数域和伸缩就好了。

\begin{prop}
  1. 任意吸收集包含零点,但$\{ \theta \}$不吸收。(因为伸缩系数为正)

  2. 任意零点的邻域是吸收的。(零点的邻域必然包含某个$B(0, \varepsilon)$,伸缩系数取$\varepsilon$(不是零!)即可,这里使用了数乘的连续性)反之并不成立,反例:$[-2, -1) \cap \{0\} \cap (1, 2]$。

  3. 若S平衡,则S吸收 iff $\forall x \in X, \exists \alpha \neq 0, \alpha x \in S$.(平衡所以不用考虑旋转,只要存在一致的伸缩系数就好了)

  4. 平衡集的闭包是平衡的。

  5. 若平衡集的内核包含零点,则内核是平衡的。(pf by def,内核是最大开集)

  6. 凸集的内核和闭包都是凸的。

  7. 吸收集/平衡集/凸集的线性组合仍是吸收集/平衡集/凸集。

  8. $\alpha, \beta \ge 0$,S凸$\Rightarrow$ $\alpha S + \beta S = (\alpha + \beta)S$。(证相互包含)
\end{prop}

\begin{thm}
  TLS上存在“好”的邻域基$\mathcal{U}$,满足:

  1. $\forall U \in \mathcal{U}$吸收且平衡。

  2. 裂变:$\forall U \in \mathcal{U}, \exists V 
  \in \mathcal{U}$ s.t. $V + V \subset U$.

  3. $\forall U_1, U_2 \in \mathcal{U}, \exists U_3 \in \mathcal{U}$ s.t. $U_3 \subset U_1 \cap U_2$.
  
  反之,若线性空间X的子集族$\mathcal{V}$满足以上三条,则存在唯一X上的拓扑使得X构成TLS,并且$\mathcal{V}$为邻域基。这说明以上三条完全刻画了X的拓扑。
\end{thm}

\begin{pf}
  $\Rightarrow$ 令$\mathcal{U} = \{ \bar{U}^b | U \in \mathcal{N}(\theta) \}$。

  0. 先证$\mathcal{U}$确实一个邻域基,即证零点的任意邻域包含更小的平衡邻域,需要使用数乘的连续性。

  1. 吸收:包含吸收集的集合吸收。平衡:显然。

  2. 裂变:使用加法的连续性,$\exists U_1, U_2 \in \mathcal{U}$ s.t. $U_1 + U_2 \subset U$,令$V = U_1 \cap U_2$即可,要证V是平衡的。

  3. 使用数乘的连续性,令$W = U_1 + U_2$, $V = \cup_{\lambda \le \delta} \lambda W$,先证$W \in \mathcal{U}$,再证V是平衡的。(其实直接由邻域基就可以得到)

  $\Leftarrow$ 令$\mathcal{N}(x) = \{ V \subset X | \exists U \in \mathcal{U} \text{ s.t. } x + U \subset V \}$,S为开集($S \in \mathcal{O}(X)$)若$\forall s \in S, S \in \mathcal{N}(s)$.

  1. 证拓扑空间,即验证三条:空集和全集,任意并封闭,有限交(两个交)封闭。

  2. 证TLS,即证加法和数乘连续。(TLS上的线性算子,在一点处连续等价于连续等价于一致连续。但是这里就是要证TLS啊,还是得证任意一点)

  加法:$m: X \times X \to X, m(x, y) = x + y$,证在$(x, y)$处连续即证$\forall U \in \mathcal{U}, \exists U_1, U_2 \in \mathcal{U}$ s.t. $x + U_1 + y + U_2 \subset x + y + U$,由裂变令$U_1 = U_2 = V$即可。

  数乘:$p: \mathbb{K} \times X \to X, p(\alpha, x) = \alpha x$,证在$(\alpha, x)$处连续即证$\forall U \in \mathcal{U}, \exists \delta > 0, U_1 \in \mathcal{U}$ s.t. $\forall |\lambda| \le \delta, (\alpha + \lambda)(x + U_1) \subset \alpha x + U$。先用裂变和平衡证一个引理:$\forall U \in \mathcal{U}, \alpha \neq 0, \exists V \in \mathcal{U}$ s.t. $\alpha V \subset U$,再用裂变,让加一项减一项之后得到的两部分都在裂变的V里面。

  3. 证邻域基,由构造过程显然。

  4. 证(拓扑的)唯一性,拓扑由邻域基唯一确定。
\end{pf}

\begin{conc}
  \textbf{(Week 1)} TLS,邻域基,abc,TLS上存在好的邻域基:平衡吸收,裂变,交性质,邻域基决定拓扑。总之是:什么是TLS?
\end{conc}

关键词:球,弱,界。

\subsection{局部凸空间}

若S是凸或平衡凸的,则S的内核以及闭包都是凸或平衡凸的。

pf: 把内核和闭包理解为最大开集和最小闭集,由定义即可。

注:abc是线性性质,内核和闭包是拓扑性质,所以这里X是TLS。

\begin{thm}
  TLS X,若U是凸邻域,则U包含一个开的平衡凸的邻域基,即凸邻域能被开的平衡凸的邻域基(平衡吸收,裂变,交性质)生成。这样在讨论局部凸空间的时候,我们的邻域基就更好了。
\end{thm}

回到那个问题:什么时候TLS能成为NLS?我们已经知道了什么是TLS,那么什么时候能成为NLS,这时候需要局部凸空间。此时,邻域基有更好的性质,更像一个NLS的邻域基,也就是单位\textbf{球}。

这里邻域基和邻域基中的元素就按照语意自己区分一下吧。

pf: 取邻域基$V \subset U$,则$\tilde{V} = (\bar{V}^c)^{\circ}$即为所求。要证$\tilde{V}$仍为邻域基(by def),以及$\tilde{V} \subset U$(凸包是最小凸集)。

局部凸空间(LCS):若邻域基中元素全是凸集。

对于球,我们想要知道其半径,于是引入“尺子”。

LS X上的半范数$p: X \to \mathbb{R}$ if

1. 正性:$p(x) \le 0$。

2. 齐性:$p(\lambda x) = |\lambda| p(x)$。

3. 三角不等式:$p(x + y) \le p(x) + p(y)$。

范数 if 定。

由于p是从X到$\mathbb{R}$的映射,我们可以将X中的那些类球邻域和真正的球进行比较。令$V_1 = p^{-1}([0, 1)) = \{x \in X | p(x) < 1\}, V_2 = p^{-1}([0, 1]) = \{x \in X | p(x) \le 1\}$,则$V_1, V_2$都是abc的。(若$X = \mathbb{R}^n$,则$V_1 = B(0, 1), V_2 = D(0, 1)$)

pf: ab由齐性,c由三角不等式。

TLS X,有了拓扑,我们讨论p的连续性:p连续 iff $V_2 \in \mathcal{N}(\theta)$。

pf: $\Rightarrow$ 显然。 $\Leftarrow$ by def: $\forall \varepsilon > 0, \exists \frac{1}{2} \varepsilon N_2 \in \mathcal{N}(\theta)$ s.t. $p(x) \le \frac{1}{2} \varepsilon < \varepsilon$. (这里是说在原点处连续,在任意一点处类似)

\begin{thm}
  LS X,$\mathcal{P}$是其上的一族半范数,令$V(p) = \{x \in X | p(x) < 1\}$,
  \[
    \mathcal{U} = \left\lbrace \bigcap_{i = 1}^n \gamma_i V(p_i) \bigg| n \in \mathbb{N}, \gamma_i \in \mathbb{K}, p_i \in \mathcal{P} \right\rbrace, 
  \]
  则$\mathcal{U}$是使得X成为LCS的邻域基(其中元素都是凸集),并且其诱导的拓扑$\tau$是使得所有p都连续的最小拓扑(最小拓扑是说,开集的个数最少)。
\end{thm}

pf: 1. 证$\mathcal{U}$是X上的邻域基:平衡吸收和分裂由$V(p)$的abc引理,交性质由构造(伸缩的有限交)。

2. 证$p$关于$\tau$连续:由上述连续性引理。

3. 证最小拓扑:设$\hat{\tau}$是使得X为LCS和所有p都连续的拓扑,则$V(p) \in \mathcal{N}_{\hat{\tau}}(\theta)$,于是其伸缩的有限交也属于后者,即$\mathcal{U} \subset \mathcal{N}_{\hat{\tau}}(\theta)$。

记$\tau := \langle \mathcal{P} \rangle$为由$\mathcal{P}$生成的局部凸拓扑。

$(X, \langle \mathcal{P} \rangle)$,则X是$T_2$的 iff $\forall x \in X, \exists p \in \mathcal{P}$ s.t. $p(x) \neq 0$(称可分点的).

pf: $\Rightarrow$ 由构造邻域基一定长成伸缩有限交的样子,$x \notin \gamma_i V(p_i) \Leftrightarrow p_i(x) > \gamma_i$ $\Leftarrow$ 显然。

LCS X, $(Y, \langle \mathcal{P} \rangle)$,则$T: X \to Y$连续 iff $\forall p \in \mathcal{P}, p \circ T: X \to \mathbb{R}$连续。

pf is omitted.(见刘培德p25定理4.5)

例子:1. 非空集合T,X为T上有界函数全体,$p_t(x) = |x(t)|$为X上的半范数,$\mathcal{P} = \{p_t, t \in T\} \Rightarrow \tau$,则$x_n \overset{\tau}{\to} x \Leftrightarrow \forall t \in T, |x_n(t) - x(t)| \to 0$,即逐点收敛。

2. TS T,X为T上连续函数全体,K为T中紧集,$p_K(x) = \max_{t \in K} |x(t)|$为X上的半范数,$\mathcal{P} = \{p_t, t \in T\} \Rightarrow \tau$,则$x_n \overset{\tau}{\to} x \Leftrightarrow \forall K \overset{compact}{\subset} T, \max_{t \in K} |x_n(t) - x(t)| \to 0$,即一致收敛。

3. B(H)上的弱算子拓扑和强算子拓扑?(博士资格考试)

\textbf{弱}:弱拓扑的弱体现在,由一族半范数生成的局部凸拓扑是最小的。(见夏道行p147定义3.3.7)

给定一族半范数可以诱导TLS上的局部凸拓扑,反过来由LCS上的拓扑能否构造一族半范数,使得其诱导的局部凸拓扑就是原来的拓扑?其难点在于如何构造这样的半范数族。

对于LS X上的ac集A,定义其上的Minkowski函数$p_A(x) = \inf \{ \lambda, x \in \lambda A \}, \forall x \in X$。若A还是b集,则$p_A$为X上的半范数。

pf is omitted.(见刘培德p5定理1.1)

对于上述ac集A,定义$A_1 = p_A^{-1}([0, 1)) = \{x \in X | p(x) < 1\}, A_2 = p_A^{-1}([0, 1]) = \{x \in X | p(x) \le 1\}$,则有$A^{\circ} \subset A_1 \subset A \subset A_2 \subset \bar{A}$。并且$p_A$连续蕴含着$\theta \in A^{\circ}$以及$A^{\circ} = A_1, A_2 = \bar{A}$。若A还是b集,则三者等价。(下面半范数族的连续性由此而来)

pf is omitted.(见刘培德p21引理4.3)

于是LCS X上的邻域基(其中元素都是abc集)上的Minkowski函数族即为想要的半范数族。这个半范数族又会诱导新的拓扑,两种拓扑的等价性证明见刘培德p23定理4.3注2。

(d). 有界集:$\forall U \in \mathcal{N}(\theta), \exists \lambda > 0$ s.t. $S \subset \lambda U$.

有界集的刻画:$S \subset X$有界 iff $\forall \{x_n\} \in S, \alpha_n \to 0(\mathbb{K})$, we have $\alpha_n x_n \to \theta(X)$.

pf is omitted.(见刘培德p13定理3.1)

可赋范化空间:若TS X上存在范数,使得该范数诱导的拓扑即为原来的拓扑。

Kolmogorov准则:TLS X是可赋范化的 iff X是T2的,且邻域基中元素是凸有界的。

\begin{conc}
  \textbf{(Week 2)} LCS,半范数,半范数族可以生成局部凸拓扑,弱拓扑,反过来LCS上也可以构造一族半范数,即邻域基元素上的Minkowski函数族,使得其诱导的局部凸拓扑就是原来的拓扑,Minkowski函数,有界集,Kolmogorov准则。

  还是那个问题:TLS什么时候是NLS?第一周讲的是什么是TLS,其由邻域基完全刻画,所以去研究邻域基。第二周讲的LCS,此时邻域基中的元素性质更好,更像是一个球。LCS由其上的一族半范数完全刻画,所以去研究半范数。还差一点有界性,TLS就是NLS了。
\end{conc}

球:形状:abc,数(半径):Minkowski函数。

弱拓扑:局部凸拓扑,使得一族半范都连续的最小拓扑

Kolmogorov准则:TLS X是可赋范化的 iff X是T2的,且存在凸有界的邻域基。

pf: 1. $\Leftarrow$ 令$U \in \mathcal{N}(\theta)$凸有界,$\exists V \subset U$平衡凸开有界,Minkowski函数$p_V$连续,为半范数。

2. 由U有界,$\forall W \in \mathcal{N}(\theta), \exists \alpha \ge 0$ s.t. $V \subset U \subset \alpha W \Rightarrow \{ \lambda V, \lambda \ge 0 \}$构成邻域基,此时$\langle p_V \rangle$生成的拓扑就是$\tau$。

3. 由T2,$x \neq \theta \Rightarrow p_V(x) \neq 0$,故$p_V$就是X上的范数。

另外TLS X什么时候是可度量化的,需要T1和邻域基可数。

一个例子:$X = C^{\infty}[0, 1], p_{n}(f) = \max_{t \in [0, 1]} |f^{(n)}(t)|$,$\{p_{n}\}$是一族半范数,X是可度量化的(可数),但不是可赋范化的。

\section{谱测度与谱分解}

\subsection{谱测度与谱积分}

分割求和取极限:$\left\{\begin{array}{ll}
  A = \sum_{i = 1}^{\infty} \lambda_n P_n \text{ (求和在强算子拓扑意义下收敛)} \\
  I = \sum_{i = 1}^{\infty} P_n
\end{array}\right.$

\subsubsection{谱测度}

两个空间:Hilbert空间H和集合X。$\mathcal{P}$是H上的投影算子族,$\mathcal{M}$是X上的$\sigma$-代数,映射$E: \mathcal{M} \to \mathcal{P}$,若

1. $E(X) = I$.

2. 可列可加性:$E(\cup_i A_i) = \sum_i E(A_i)$,$A_i \in \mathcal{M}$两两不交。

则称E为$(X, \mathcal{M})$上(H中)的谱测度,$(X, \mathcal{M}, E)$为H中的谱测度空间。

谱测度的性质:

1. 若$A, B \in \mathcal{M}$不交,则$E(A) E(B)=E(B) E(A)=0$。

2. 若$A, B \in \mathcal{M}$,则$E(A \cup B)=E(A)+E(B)-E(A \bigcap B)$(作为投影算子)。

3. $\{E(A) \mid A \in \mathcal{M}\}$是交换算子族.

谱测度诱导$(X, \mathcal{M})$上的测度:

1. $\mu_x(A) = \langle E(A)x, x \rangle$.

2. $\mu_{x, y}(A) = \langle E(A)x, y \rangle$.

这里$x, y \in H$是给定的,所以是不确定的,由此导出不同谱积分的定义。

用$\mathcal{M}(X)$表示X上可测函数全体,$B(X, \mathcal{M})$表示X上有界可测函数全体。

\subsubsection{谱积分}

定义$f \in B(X, \mathcal{M}), f: X \to \mathbb{K}$关于谱测度E的弱谱积分$T: B(X, \mathcal{M}) \to \mathcal{P}, T[f] = \int_X f(t) E(dt)$ s.t.
\[
  \langle \int_X f(t) E(dt) x, y \rangle = \int_X f(t) \langle E(dt) x, y \rangle
\]
i.e.,
\[
  \langle I[f]x, y \rangle = \int_X f(t) \mu_{x, y}(dt)
\]

弱谱积分的性质:

1. 1.5线性。

2. 压缩性:$\|I[f]\|_{H'} \le \|f\|_{\infty}$。

3. $E(A) = \int_X 1_A(t) E(dt)$.

pf: 2. 定义双线性泛函$\Phi(x, y) = \langle I[f]x, y \rangle$,对$\Phi(x, x)$用Riesz表示定理。

一致谱积分:按照值域分割求和取极限的过程定义。

两种谱积分本质上是等价的。

第一种定义比较简单,第二种是按照值域划分的过程来定义的,主要用于证明(建立积分)。

$\forall f, g \in B(X, \mathcal{M})$,$T[f]$与$T[g]$(作为投影算子)可交换(可换族)。

pf: 上面谱测度性质第三条相当于是说对于f和g是示性函数的时候上式成立,然后再按照测度论里面的思路,先对一个示性,简单,有界可测,然后再对另一个做就好了。(这里最大的空间就是有界可测函数就完了,没有一般可测)

$\forall f, g \in B(X, \mathcal{M})$,$T[f \cdot g] = T[f] \cdot T[g]$,乘法理解为函数乘法和投影算子复合。

pf: 示性,简单(线性),有界可测(控制收敛)。

\subsection{谱系}

$\mathbb{R}$上的测度与$\mathbb{R}$上单调递增的右连续函数一一对应。

$X = \mathbb{R}$上H中的谱测度与$X = \mathbb{R}$上单调递增的右连续投影算子值函数(谱系)一一对应。

\begin{df}(谱系)
  Hilbert空间H,$\{E_\lambda\}_{\lambda \in \mathbb{R}}$是一族投影算子,若有
  \begin{enumerate}
    \item 单调性:$\forall \lambda, \mu \in \mathbb{R}, \lambda \ge \mu$,有$E_{\lambda} \ge E_{\mu}$。(投影算子值域空间的单调性,越来越大)
    \item 右连续性:$\forall \lambda \in \mathbb{R}$,有(强)$E_{\lambda + 0} = E_\lambda$。
    \item 强算子拓扑收敛:(强)$\lim_{\lambda \to -\infty} E_{\lambda}=0$,(强)$\lim_{\lambda \rightarrow+\infty} E_{\lambda}=I$。
  \end{enumerate}
\end{df}

定义中的强收敛可以弱化为弱收敛。

谱系的充要条件(见夏道行实变与泛函下p287定理6.7.3$'$)

一个例子:$E_{\lambda} f=1_{(-\infty, \lambda]}(t) f(t), f \in H = L^{2}[0,1]$。

\subsection{有界自伴算子谱分解}

有界变差空间$BV[a, b] = \{g: [a, b] \to \mathbb{R} \mid g(a) = 0, g \text{ 右连续 }, \operatorname{Var}(g) < \infty\}$,其中,1. 左端点函数值为零和右连续称为规范化,是为了之后定义范数时的正定性。2. $\operatorname{Var}(g) = \sup_{\Delta} \operatorname{Var}(g, \Delta)$. 3. 有界变差则几乎处处可导。

\begin{lem}
  $C[0, 1]^{\ast} \cong BV[0, 1]$.
\end{lem}

pf: 令$\pi: BV[0, 1] \to C[0, 1]^{\ast}, \pi(g)(f) = \int_0^1 f(t) dg(t) = \int_0^1 f(t) g'(t) dt$(测度加权还是测度),则$\pi$是$BV[0, 1]$到$C[0, 1]^{\ast}$的等距同构。

\begin{lem}(唯一性引理)
  设$v \in BV[0, 1]$,若$\int_0^1 f(t) dv(t) = 0, \forall f \in C[0, 1]$,则$v = 0$。
\end{lem}

pf: n次方,多项式(线性),连续(Weierstrass定理 + 有界线性泛函)。

\begin{lem}
  令$u(t) = \int_0^t g(s) dv(s)$,则$\int_0^1 f(t) du(t) = \int_0^1 f(t) g(t) dv(t)$。
\end{lem}

\begin{lem}
  设$A = A^{\ast} \in B(H)$为H上的自伴算子,$\langle Ax, x \rangle \in \mathbb{R}, \forall x \in H$,令$M = \sup_{\|x\| = 1}\langle Ax, x \rangle, m = \inf_{\|x\| = 1}\langle Ax, x \rangle$,则
  \begin{enumerate}
    \item $\|A\| = \max\{|m|, |M|\}$.
    \item $\sigma(A) \in [m, M]$.
    \item 谱半径$r_\sigma(A) := \sup_{\lambda \in \sigma(A)} |\lambda|$,有$r_\sigma(A) = \|A\|$。
  \end{enumerate}
\end{lem}

\begin{lem}(谱映射引理)
  设$p(\cdot)$为抽象多项式,则$\sigma(p(A)) = p(\sigma(A))$,即(算子)多项式的谱等于谱的(数值)多项式。
\end{lem}

pf by 代数基本定理。

\begin{thm}(有界自伴算子谱分解定理)

  给定$A = A^{\ast} \in B(H)$为H上的自伴算子,存在谱系$\{E_\lambda\}_{\lambda \in \mathbb{R}}$ s.t.
  \begin{enumerate}
    \item 投影性质:$\lambda \le \mu, E_\lambda E_\mu = E_\mu E_\lambda = E_\lambda$。
    \item 右连续性:$\forall \lambda \in \mathbb{R}$,有(强)$E_{\lambda + 0} = E_\lambda$。
    \item $E_\lambda = 0, \lambda < m; E_\lambda = I, \lambda \ge M$.
    \item 交换性:$E_\lambda A = A E_\lambda$。
  \end{enumerate}
  进而$f(\lambda) = \langle E_\lambda x, y \rangle$有界变差,且$\langle p(A) x, y \rangle = \int_m^M p(\lambda) d\langle E_\lambda x, y \rangle$,其中$p(\cdot)$为抽象多项式。
\end{thm}

由有界自伴算子能找到谱系。

\begin{pf}
  \quad
  \begin{enumerate}
    \item $[m, M]$上的多项式函数空间$P[m, M]$稠于连续函数空间$C[m, M]$。
    \item 固定$x, y \in H$,令$L(p) = \langle p(A)x, y \rangle$,则L是$P[m, M]$上的有界线性泛函,可以延拓到$C[m, M]$上,即$L \in C[m, M]^{\ast}$,注意这里用连续性和稠密性即可,不用Hahn-Banach延拓定理。
    \item 由引理一,$\exists V(\lambda;x, y) \in BV[m, M]$ s.t. $\pi(V) = L$ i.e. $\pi(V)(f) = L(f)$,在$P[m, M]$上即$\int_m^M f dV(\lambda;x, y) = \int_m^M f d\langle p(A)x, y \rangle$。允许$x, y \in H$动起来,则$V(\lambda;x, y)$是三变量函数。(“可进行规范化,确保唯一性”)
    \item 固定$\lambda$,$V(\lambda;x, y)$是有界共轭双线型且Hermite。共轭双线性由唯一性引理得到,有界性直接放到最大值就好了,Hermite略。
    \item 由$V(\lambda;x, y)$找到H上的自伴算子$\langle F_\lambda x, y \rangle = V(\lambda;x, y)$。下证$\{F_\lambda\}$为(我们要找的)谱系。
    \item 最主要是证投影性质,先对n次方函数证明(需要费一番周折),由唯一性引理可得。最后令$E_\lambda = F_\lambda$即可。
  \end{enumerate}
\end{pf}

\begin{rmk} \quad

  \begin{enumerate}
    \item $p(A) = \int_m^M p(\lambda) dE_\lambda$一致成立,即在强算子拓扑下成立。要想把p推广到有界可测函数f,需要控制收敛定理,这由有限测度和有界函数直接得到。
    \item 这里的积分限可以是任意$[\alpha, \beta] \supset [m, M]$,都写$[m, M]$是因为我当时没想好这么解释。
  \end{enumerate}
\end{rmk}

\begin{thm}(真正的有界自伴算子谱分解定理,提升至算子代数)

  给定$A = A^{\ast} \in B(H)$为H上的自伴算子,则$\pi: C[m, M] \to B(H), \pi(f) = f(A) = \int_m^M f(\lambda) dE_\lambda$ s.t.
  \begin{enumerate}
    \item 线性。
    \item 代数同态:$\pi(f \cdot g) = \pi(f) \cdot \pi(g)$。(Banach代数)
    \item 若B与$E_\lambda$可交换,则与$f(A)$可交换。
    \item $f(A)$正规,即$f(A)^{\ast}f(A) = f(A) f(A)^{\ast}$,进一步若f实,则$f(A)$自伴。
    \item $\pi$压缩,即$\|f(A)\| \le \|f\|_{\infty}$。
    \item $\|f(A)x\|^2 = \int_m^M |f(\lambda)|^2 d \|E_\lambda x\|^2$.
  \end{enumerate}
\end{thm}

\begin{thm}
  设$\lambda_0 \in \mathbb{R}$,则$\lambda_0 \in \rho(A)$ iff $\exists \varepsilon > 0$ s.t. $E_\lambda$在$[\lambda_0 - \varepsilon, \lambda_0 + \varepsilon]$上为常值。
\end{thm}

\begin{rmk}
  对于自伴算子$A = A^{\ast}$,
  \begin{enumerate}
    \item 设$\lambda \in \mathbb{C}$,若$\operatorname{Im}\lambda \neq 0$,则$\lambda \in \rho(A)$。
    \item 设$\lambda \in \mathbb{R}$,若$|\lambda| \ge r_\sigma(A) = \|A\|$,则$\lambda \in \rho(A)$。(什么幂级数展开)
  \end{enumerate}
\end{rmk}

\begin{pf}
  $\Leftarrow$ 令$f(\lambda) = \lambda_0 - \lambda, g(\lambda) = \left\{ \begin{array}{ll}
    \frac{1}{\lambda_0 - \lambda}, \quad \lambda \notin [\lambda_0 - \varepsilon, \lambda_0 + \varepsilon],\\
    \text{连续}, \quad \lambda \in [\lambda_0 - \varepsilon, \lambda_0 + \varepsilon].
  \end{array} \right.$则$\pi[f \cdot g] = f(A) \cdot g(A) = \int_m^M f(\lambda) \cdot g(\lambda) dE_\lambda = \int_m^{\lambda_0 - \varepsilon} + \int_{\lambda_0 - \varepsilon}^{\lambda_0 + \varepsilon} + \int_{\lambda_0 + \varepsilon}^M = 0 + I(A)$,即$g(A) = f^{-1}(A) = (\lambda_0 I - A)^{-1}$,故$\lambda_0 \in \rho(A)$。

  $\Rightarrow$ 反证法。若不是常数,则$\forall \varepsilon, \exists \lambda_0 - \varepsilon \le \lambda_1 < \lambda_2 \le \lambda_0 + \varepsilon, E_{\lambda_1} < E_{\lambda_2}$,谱系是一族投影算子,投影算子的大小是说其值域的大小,而投影算子的值域是H的线性子空间,仍为Hilbert空间,于是存在$y \in E_{\lambda_2}, \|y\| = 1, y \perp E_{\lambda_1}$,则$E_\lambda y = \left\{ \begin{array}{ll}
    y, \quad \lambda > \lambda_2,\\
    0, \quad \lambda < \lambda_1.
  \end{array} \right.$于是$\|(\lambda_0 I - A)y\|^2 = \int_m^M |\lambda_0 - \lambda|^2 d \|E_\lambda y\|^2 = \int_{\lambda_1}^{\lambda_2} |\lambda_0 - \lambda|^2 d \|E_\lambda y\|^2 \le \varepsilon^2 \|y\|^2 = \varepsilon^2$,则$\inf_{\|y\| = 1} \|(\lambda_0 I - A)y\| = 0$,这可以说明$\lambda_0 \in \sigma(A)$,矛盾!

  其中最后一句话是因为,若$\lambda_0 \in \rho(A)$,则$(\lambda_0 I - A)^{-1}$存在且有界。由$\inf_{\|y\| = 1} \|(\lambda_0 I - A)y\| = 0$,$\forall k \in \mathbb{N}, \exists y_k \in H$ s.t. $\|y_k\| = 1, \|(\lambda_0I - A)y_k\| \le \frac{1}{k}$,则$1 = \|(\lambda_0I - A)^{-1} (\lambda_0I - A)y_k\| \le \|(\lambda_0I - A)^{-1}\| \cdot \|(\lambda_0I - A)y_k\| \le \frac{1}{k} \|(\lambda_0I - A)^{-1}\|$,与其有界性矛盾。
\end{pf}

自伴算子的谱在实轴上,酉算子的谱在单位圆上,二者是代数同构,因此酉算子的谱也是“可演算的”,区别在于自伴算子的起点是多项式函数空间,酉算子的起点是三角多项式函数空间。

一些空间和映射:

1. Hilbert空间H

2. 集合X

3. $\mathcal{P} \subset B(H)$是H上的投影算子族($B(H)$的线性子空间),$P \in \mathcal{P}, P: H \to H$

4. $\mathcal{M}$是X上的$\sigma$-代数

5. X上H中的谱测度$E: \mathcal{M} \to \mathcal{P}$

6. 固定$x, y \in H$,X上的测度$\mu_x: \mathcal{M} \to \mathbb{K}, \mu_x(A) = \langle E(A)x, x \rangle; \mu_{x, y}: \mathcal{M} \to \mathbb{K}, \mu_{x, y}(A) = \langle E(A)x, y \rangle$

7. 有界可测函数$f \in B(X, \mathcal{M}), f: X \to \mathbb{K}$

8. 谱积分$T: B(X, \mathcal{M}) \to \mathcal{P}, T[f] = \int_X f(t) E(dt)$

9. 谱系$\{E_\lambda\}_{\lambda \in \mathbb{R}} \subset \mathcal{P}$,硬要说就是$E: \mathbb{R} \to \mathcal{P}$(这里$X = \mathbb{R}$,一般集合谈不了左右)

谱的定义

若$(\lambda - A)^{-1}$存在且有界,则$\lambda \in \rho(A)$称正则点,$\rho(A)$称A的预解集,$(\lambda - A)^{-1}$称A的预解式。

若否,则$(\lambda - A)^{-1}$不存在或存在但无界,此时$\lambda \in \sigma(A) = \mathbb{C} - \rho(A)$称谱点。

若不存在,说明$\lambda - A$不单,或Ker$(\lambda - A)$有维数,此时$\lambda \in \sigma_p(A)$称特征值。

若存在但无界,再若稠定,此时$\lambda \in \sigma_c(A)$称连续谱。若不稠定,此时$\lambda \in \sigma_r(A)$称剩余谱。 

\section{广义函数}

与PDE里面那个过程是一样的,就是可积函数由内积可以看成是有界线性泛函,但是有界线性泛函不一定是可积函数,所以引出一类新的“函数”,即广义函数。

空间的对偶关系:$(L^p[a, b])^{\ast} = L^q[a, b], (C[a, b])^{\ast} = BV[a, b]$。

\subsection{基本函数空间}

“广义函数是基本空间上的有界线性泛函”

\begin{df}(基本空间与拓扑(收敛性))

  \begin{enumerate}[(1).]
    \item $\mathscr{E}(\Omega) = \mathscr{E}(\Omega)$:(收敛性)$\varphi_{\nu} \rightarrow 0\left(C^{\infty}\left(\Omega\right)\right)$ if $\sup _{x \in K}\left|\partial^{\alpha} \varphi_{\nu}\right| \rightarrow 0$。(这里K是任意紧集)
    \item $\mathscr{D}(\Omega) = C^{\infty}_0(\Omega)$:(收敛性)$\varphi_{\nu} \rightarrow 0\left(C^{\infty}_0\left(\Omega\right)\right)$ if $\sup _{x \in K}\left|\partial^{\alpha} \varphi_{\nu}\right| \rightarrow 0$。(这里要求$\varphi_{\nu}$的支集包含在同一个紧集中,即K)
    \item $\mathscr{P}(\Omega)$:(定义)首先要求函数$u \in \mathscr{E}(\Omega)$,其次满足速降条件,其有三个等价描述:
    \begin{enumerate}[1.]
      \item $\forall \alpha, p, \lim _{|x| \rightarrow \infty} x^{\alpha} \partial^{p} \varphi(x)=0$。
      \item $\forall \alpha, p, x^{\alpha} \partial^{p} \varphi(x)$在$\Omega$上有界。
      \item $\forall k, p, (1 + |x|^2)^k \partial^{p} \varphi(x)$在$\Omega$上有界。
    \end{enumerate}
    (收敛性)$\varphi_{\nu} \rightarrow 0\left(\mathscr{P}\left(\Omega\right)\right)$ if $\sup_{x \in \Omega}\left|x^{\alpha} \partial^{p} \varphi_{\nu}(x)\right| \rightarrow 0, $。
  \end{enumerate}
\end{df}

\begin{rmk}
  \begin{enumerate}
    \item 基本函数空间上甚至没有度量,是一个拓扑线性空间,进一步是局部凸空间,其上的半范数族为$\mathcal{P} = \{ p_{K, \alpha}(\varphi) = \sup_{x \in K}\left|\partial^{\alpha} \varphi(x) \right|\}$。
    \item 基本函数空间是完备的。完备是拓扑概念,因为Cauchy列和收敛列都可以用拓扑定义。
  \end{enumerate}
\end{rmk}

\begin{df}
  磨光算子:$\varphi(x)=\left\{\begin{array}{ll}
    e^{\frac{1}{|x|^{2}-1}}, & |x|<1 \\
    0, & |x| \geqslant 1
  \end{array}\right.$,$\alpha(x)=\frac{1}{\int_{\mathbb{R}^n} \varphi(x) dx} \varphi(x)$是其单位化,$\alpha_{\varepsilon}(x)=\frac{1}{\varepsilon^{n}} \alpha\left(\frac{x}{\varepsilon}\right)$。令$J_{\epsilon} u = u_{\epsilon} = u * a_{\epsilon}$,则$J_{\epsilon}$称为磨光算子,即有(若u局部可积)$J_{\epsilon} u \in C^{\infty}$。
\end{df}

截断函数略。

\begin{prop}(空间包含关系)

  基本空间有:$\mathscr{E}(\mathbb{R}^n) \supset \mathscr{P}(\mathbb{R}^n) \supset \mathscr{D}(\mathbb{R}^n)$。
  
  广义函数空间有:$\mathscr{E}'(\mathbb{R}^n) \subset \mathscr{P}'(\mathbb{R}^n) \subset \mathscr{D}'(\mathbb{R}^n)$。

  另外:$L_{loc}'(\Omega) \subset \mathscr{D}'(\Omega)$。
\end{prop}

证明即由定义证有界线性泛函,并不是平凡的。

\subsection{广义函数及其运算}

\begin{thm}(广义函数的刻画)

  \begin{enumerate}
    \item 若$T \in \mathscr{D}'(\Omega)$,则对任一紧集$K \subset \Omega$,存在常数$C(K) > 0$和非负整数$m(K)$,使得
    \[
      |\langle T, \varphi\rangle| \leqslant C \sup _{\substack{x \in \Omega \\|\alpha| \leqslant m}}\left|\partial^{\alpha} \varphi(x)\right|, \forall \varphi \text{ s.t. supp} \varphi \subset K.
    \]
    反之,若T为$\mathscr{D}(\Omega)$上的线性泛函,且上式成立,则$T \in \mathscr{D}'(\Omega)$。
    \item 若$T \in \mathscr{E}'(\Omega)$,则存在紧集$K \subset \Omega$、常数$C > 0$和非负整数$m$,使得
    \[
      |\langle T, \varphi\rangle| \leqslant C \sup _{\substack{x \in K \\|\alpha| \leqslant m}}\left|\partial^{\alpha} \varphi(x)\right|, \forall \varphi \in \mathscr{E}(\Omega).
    \]
    反之,若T为$\mathscr{E}(\Omega)$上的线性泛函,且上式成立,则$T \in \mathscr{E}'(\Omega)$。
    \item 若$T \in (\mathscr{P})'(\mathbb{R}^n)$,则存在常数$C > 0$和非负整数$m, N$,使得
    \[
      |\langle T, \varphi\rangle| \leqslant C \sum_{|\alpha| \leqslant m} \sup _{\substack{x \in \mathbb{R}^n}} {(1 + |x|^2)}^N \left|\partial^{\alpha} \varphi(x)\right|, \forall \varphi \in \mathscr{P}(\mathbb{R}^n).
    \]
    反之,若T为$\mathscr{P}(\mathbb{R}^n)$上的线性泛函,且上式成立,则$T \in (\mathscr{P})'(\mathbb{R}^n)$。
  \end{enumerate}
\end{thm}

广义函数的性质

\begin{df}
  广义函数的支集(讨论广义函数在一点的取值是没有意义的,因为Lebesgue可积函数允许在一个零测集上改变取值,但是广义函数在一个开集上的值是可以定义的):设$T \in \mathscr{D}'(\Omega)$,称T在$\Omega' \subset \Omega$内为0,若$\langle T, \varphi \rangle = 0, \forall \varphi \in \mathscr{D}(\Omega')$。广义函数的支集定义为取零值的最大开集的余集。
\end{df}

\begin{thm}
  设$T \in \mathscr{D}'(\Omega)$,则$T \in \mathscr{E}'(\Omega)$ iff T紧支。
\end{thm}

\begin{df}
  广义函数的极限:$T_k$弱收敛于0,若$\langle T_k, \varphi \rangle \to 0, \forall \varphi$。

  弱极限与弱$\ast$极限:
  \begin{enumerate}
    \item Let $x_n, x \in X$, then $x_n \overset{w}{\longrightarrow} x$ if $f(x_n) \longrightarrow f(x), \forall f \in X^{\ast}$.
    \item Let $f_n, f \in X^{\ast}$, then $f_n \overset{w-\ast}{\longrightarrow} f$ if $f_n(x) \longrightarrow f(x), \forall x \in X$.
  \end{enumerate}
\end{df}

\begin{df}
  广义函数的导数:$\left\langle\frac{\partial T}{\partial x_{k}}, \varphi\right\rangle = -\left\langle T, \frac{\partial \varphi}{\partial x_{k}}\right\rangle, \forall \varphi \in \mathscr{D}\left(\mathbb{R}^{n}\right)$以及高阶导数$\left\langle\partial^{\alpha} T, \varphi\right\rangle=(-1)^{|\alpha|}\left\langle T, \partial^{\alpha} \varphi\right\rangle, \forall \varphi \in \mathscr{D}\left(\mathbb{R}^{n}\right)$。广义函数无穷阶可导并且求导次序可交换。
\end{df}

\begin{df}
  广义函数的卷积:(形式定义)$\langle S * T, \varphi\rangle=\left\langle S_{x},\left\langle T_{y}, \varphi(x+y)\right\rangle\right\rangle, \forall \varphi \in \mathscr{D}\left(\mathbb{R}^{n}\right)$。S和T中至少有一个紧支,则$S * T \in \mathscr{D}'(\mathbb{R}^{n})$
\end{df}

\begin{prop}(卷积的性质)
  \begin{enumerate}
    \item 结合律,交换律
    \item 单位元为Dirac函数
    \item $\partial^\alpha(S * T) = (\partial^\alpha S) * T = S * (\partial^\alpha T)$
  \end{enumerate}
\end{prop}

广义函数的Fourier变换略。

\vspace{60pt}





















